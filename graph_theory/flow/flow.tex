\subsection{网络流图}
\lstinputlisting{graph_theory/flow/flow_graph.cpp}

\subsection{Edmonds-Karp 算法}
\lstinputlisting{graph_theory/flow/edmonds_karp.cpp}

\subsection{Dinic 算法}

Dicnic算法不断重复以下过程:首先从源点沿着可增广边做一遍广搜,给每个点标记一个距离。如果遍历不到汇点,即找不到增广路,算法结束。

在增广的时候,只选择距离恰好是自己距离加1的点扩展。这样保证了每次以最短路增广。其次在找到了一条增广路后,并不是立刻回退到源点,而是寻找到增广路上第一个满流的边的起点继续增广。

\lstinputlisting{graph_theory/flow/dinic.cpp}

\subsection{ISAP 算法}
\lstinputlisting{graph_theory/flow/isap.cpp}

\subsection{费用流图}
\lstinputlisting{graph_theory/flow/cost_flow_graph.cpp}

\subsection{最小费用最大流}
\lstinputlisting{graph_theory/flow/min_cost_max_flow.cpp}
