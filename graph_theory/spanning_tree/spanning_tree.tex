\paragraph{问题} 要在$n$个城市之间铺设光缆,要求这$n$个城市的任意两个之间都可以通信,然而,铺设光缆的费用很高,且各个城市之间铺设光缆的费用不同,所以还需要使铺设光缆的总费用最低。请问最低费用是多少?

\paragraph{最小生成树} 对于一个无向图来说,最小生成树是一个这样的子图:它内部没有环,各节点之间可以 互相抵达,而且把最小生成树上各边的权值相加之后,得到的和是最小的。

\subsection{Prim}

Prim算法是贪心算法,贪心策略为:找到目前情况下能连上的权值最小的边的另一端点,加入之,直到所有的顶点加入完毕。

Prim适用于稠密图,朴素Prim的时间复杂度是$\mathcal{O}(n^2)$。因为在寻找离生成树最近的未加入顶点时浪费了很多时间,所以可以用堆进行优化。堆优化后的Prim算法的时间复杂度为$\mathcal{O}(m\log n)$。

\lstinputlisting{graph_theory/spanning_tree/prim.cpp}

\subsection{Prim (堆优化)}
\lstinputlisting{graph_theory/spanning_tree/prim_heap.cpp}

\subsection{Kruskal}

Kruskal算法是贪心算法,贪心策略为:选目前情况下能连上的权值最小的边,若与已经生成的树不会形成环,加入之,直到$n−1$条边加入完毕。

时间复杂度为$\mathcal{O}(n\log m)$,最差情况为$\mathcal{O}(m\log n)$。相比于Prim,这个算法更常用。

\lstinputlisting{graph_theory/spanning_tree/kruskal.cpp}

\subsection{次小生成树}

次小生成树可以由最小生成树换一条边得到。这样就有一种比较容易想到的算法——枚举删除最小生成树上的边,再求最小生成树,然而算法的效率并不是很高。

有一种更简单的方法:先求最小生成树$T$,枚举添加不在$T$中的边,则添加后一定会形成环。找到环上边值第二大的边(即环中属于$T$中的最大边),把它删掉,计算当前生成树的权值,取所有枚举修改的生成树的最小值,即为次小生成树。

这种方法在实现时有更简单的方法:首先求最小生成树$T$,然后从每个结点$u$遍历最小生成树$T$,用一个二维数组$f[u][v]$记录结点$u$到结点$v$的路径上边的最大值(即最大边的值)。然后枚举不在$T$中的边$(u, v)$,计算$T-f[u][v]+w(u,v)$的最小值,即为次小生成树的权值。

显然,这种方法的时间复杂度为$\mathcal{O}(N^2+E)$。

以下的模板需要先求出Prim,然后通过参数传递Prim的值,解出次小生成树。

\paragraph{例题} POJ1679

\lstinputlisting{graph_theory/spanning_tree/2nd_prim.cpp}

\subsection{最小树形图朱刘算法}

\paragraph{最小树形图} “最小生成树”针对的是无向图。换成有向图,则是“最小树形图”。

\paragraph{问题} 在有向带权图中指定一个特殊的点$v$,求一棵以$v$为根的有向生成树$T$,并且$T$中所有边的总权值最小。

\lstinputlisting{graph_theory/spanning_tree/edmonds.cpp}
