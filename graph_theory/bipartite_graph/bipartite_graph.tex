\subsection{相关概念和性质}

\paragraph{点覆盖集} 为一个点集,使得所有边至少有一个端点在集合里,或者说是“点”覆盖了所有“边”。

极小点覆盖(minimal vertex covering):本身为点覆盖,其真子集都不是。

最小点覆盖(minimum vertex covering):点最少的点覆盖。

点覆盖数(vertex covering number):最小点覆盖的点数。

\paragraph{边覆盖集} 为一个边集,使得所有点都与集合里的边邻接。或者说是“边”覆盖了所有“点”。

极小边覆盖(minimal edge covering):本身是边覆盖,其真子集都不是。

最小边覆盖(minimum edge covering):边最少的边覆盖。

边覆盖数(edge covering number):最小边覆盖的边数。

\paragraph{独立集} 为一个点集,集合中任两个结点不相邻,则称$V$为独立集。或者说是导出的子图是零图(没有边)的点集。

极大独立集(maximal independent set):本身为独立集,再加入任何点都不是。

最大独立集(maximum independent set):点最多的独立集。

独立数(independent number):最大独立集的点。

\paragraph{团} 为一个点集,集合中任两个结点相邻。或者说是导出的子图是完全图的点集。

极大团(maximal clique):本身为团,再加入任何点都不是。

最大团(maximum clique):点最多的团。

团数(clique number):最大团的点数。

\paragraph{边独立集} 为一个边集,满足边集中的任两边不邻接。

极大边独立集(maximal edge independent set):本身为边独立集,再加入任何边都不是。

最大边独立集(maximum edge independent set):边最多的边独立集。

边独立数(edge independent number):最大边独立集的边数。

边独立集又称匹配(matching),相应的有极大匹配(maximal matching),最大匹配(maximum matching),匹配数(matching number)。

\paragraph{支配集} 为一个点集,使得所有其他点至少有一个相邻点在集合里。或者说是一部分的“点”支配了所有“点”。

极小支配集(minimal dominating set):本身为支配集,其真子集都不是。

最小支配集(minimum dominating set):点最少的支配集。

支配数(dominating number):最小支配集的点数。

\paragraph{边支配集} 为一个边集,使得所有边至少有一条邻接边在集合里。或者说是一部分的“边”支配了所有“边”。

极小边支配集(minimal edge dominating set):本身是边支配集,其真子集都不是。

最小边支配集(minimum edge dominating set):边最少的边支配集。

边支配数(edge dominating number):最小边支配集的边数。

\paragraph{最小路径覆盖} 是“路径”覆盖“点”,即用尽量少的不相交简单路径覆盖有向无环图$G$的所有顶点,即每个顶点严格属于一条路径。路径的长度可能为0(单个点)。

最小路径覆盖 = $G$的点数 - 最小路径覆盖中的边数。应该使得最小路径覆盖中的边数尽量多,但是又不能让两条边在同一个顶点相交。

拆点:将每一个顶点$i$拆成两个顶点$X_i$和$Y_i$。然后根据原图中边的信息,从$X$部往$Y$部引边。所有边的方向都是由$X$部到$Y$部。因此,所转化出的二分图的最大匹配数则是原图$G$中最小路径覆盖上的边数。因此由最小路径覆盖 = 原图$G$的顶点数 - 二分图的最大匹配数便可以得解。

\subsection{二分图判断}

\paragraph{二分图} 有两顶点集且图中每条边的的两个顶点分别位于两个顶点集中,每个顶点集中没有边直接相连接。

无向图$G$为二分图的充分必要条件是,$G$至少有两个顶点,且其所有回路的长度均为偶数。

判断二分图的常见方法是\textbf{染色法}:开始对任意一未染色的顶点染色,之后判断其相邻的顶点中,若未染色则将其染上和相邻顶点不同的颜色,若已经染色且颜色和相邻顶点的颜色相同则说明不是二分图,若颜色不同则继续判断,BFS和DFS可以搞定。

\begin{itemize}
  \item 无向图$G$有$n$个顶点,且没有孤立顶点:\\
  点覆盖数 + 点独立数 = $n$ \\
  边覆盖数 + 边独立数 = $n$
  \item 对于二分图$G$有$n$个顶点,且没有孤立顶点:\\
  点覆盖数 = 匹配数 \\
  点独立数 = $n$ - 匹配数 \\
  如果二分图有$m$个孤立顶点:独立数 - $m$ = $n$ - $m$ - 匹配数
\end{itemize}

\paragraph{例题} HDU2444

\paragraph{注} DFS和BFS判断二分图的方法分别在\textbf{匈牙利算法}的DFS和BFS版本中的is\_valid函数中实现。

\subsection{Hopcroft Karp 算法}
\lstinputlisting{graph_theory/bipartite_graph/hopcroft_karp.cpp}

\subsection{Kuhn Munkras 算法}
\lstinputlisting{graph_theory/bipartite_graph/kuhn_munkras.cpp}

\subsection{匈牙利算法 (DFS)}
\lstinputlisting{graph_theory/bipartite_graph/hungary_dfs.cpp}

\subsection{匈牙利算法 (BFS)}
\lstinputlisting{graph_theory/bipartite_graph/hungary_bfs.cpp}
