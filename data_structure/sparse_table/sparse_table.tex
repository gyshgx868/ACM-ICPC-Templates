RMQ(Range Minimum/Maximum Query)问题是指:已知长度为$n$的数列$A$,需要查询某个区间的最值若干次。

由于询问次数很多,并且区间范围不确定,因此采用朴素算法是不可行的。若采用线段树,预处理的时间为$\mathcal{O}(n)$,查询的时间是$\mathcal{O}(\log n)$。并且,如果规定数列元素可以更改,那么基本上只能用线段树来处理。这种情况下,该算法不可用。

RMQ是在线算法,预处理的时间为$\mathcal{O}(n\log n)$,但回答一次询问的仅为$\mathcal{O}(1)$。

\subsection{RMQ}

\lstinputlisting{data_structure/sparse_table/sparse_table.cpp}

\paragraph{注} 可以将上述代码中的Min替换为Max或者gcd来维护区间最小值或者最大公约数,仿函数的写法如下:

\lstinputlisting{data_structure/sparse_table/operation.cpp}

\subsection{RMQ 维护区间和}
\lstinputlisting{data_structure/sparse_table/sum_sparse_table.cpp}

\subsection{2D RMQ}

\paragraph{例题} HDU2888(该题需要将vector替换为数组,否则会MLE)

\lstinputlisting{data_structure/sparse_table/sparse_table_2d.cpp}
