\section{Java读写文件}

从文件读入:

\begin{lstlisting}
Scanner in = new Scanner(new File("data.in"));
BufferedReader in = new BufferedReader(new FileReader("data.in"));
\end{lstlisting}

写入到文件:

\begin{lstlisting}
PrintWriter out = new PrintWriter("test.out");
out.println("test");
out.close();
\end{lstlisting}

\section{String}

\begin{enumerate}
  \item StringTokenizer属于java.util类,引用的时候导入类:import java.util.*;
  \item 构造函数:
    \begin{enumerate}
      \item StringTokenizer(String str) \\
            构造一个解析str的对象,这时候默认的Token间隔符有“空格”、“制表符”、“换行符”、“回车符”。
      \item StringTokenizer(String str, String delim) \\
            构造一个用来解析str的StringTokenizer对象,并提供一个指定的分隔符,如“,”等。
      \item StringTokenizer(String str, String delim, boolean returnDelims) \\
            构造一个用来解析str的StringTokenizer对象,并提供一个指定的分隔符(同2),同时,指定是否返回分隔符。
    \end{enumerate}
  \item 方法:
    \begin{enumerate}
      \item int countTokens():返回nextToken方法被调用的次数。如果采用构造函数1和2,返回的就是分隔符数量。
      \item boolean hasMoreTokens():返回是否还有分隔符。
      \item boolean hasMoreElements():结果同上。
      \item String nextToken():返回从当前位置到下一个分隔符的字符串。
      \item Object nextElement():结果同4。
      \item String nextToken(String delim):与4类似,以指定的分隔符返回结果。
    \end{enumerate}
\end{enumerate}

\begin{lstlisting}
String s = "The=Java=platform=is=the=ideal=platform=for=network=computing";
StringTokenizer st = new StringTokenizer(s, "=", true); // 当为false时不包括=
System.out.println("Token Total: " + st.countTokens());
while (st.hasMoreElements()) {
  System.out.println(st.nextToken());
}
\end{lstlisting}
