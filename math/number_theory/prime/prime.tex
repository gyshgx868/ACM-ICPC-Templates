\subsection{素数判断}

\paragraph{例题} HDU2138

\lstinputlisting{math/number_theory/prime/is_prime.cpp}

\subsection{筛法求素数}
\lstinputlisting{math/number_theory/prime/sieve_eratosthenes.cpp}

\subsection{区间筛素数}

筛选区间$[l,r)$中的素数,is\_prime[i-l]=true <=> $i$是素数(下标偏移了$l$)。

\lstinputlisting{math/number_theory/prime/sieve_interval.cpp}

\subsection{素数计数}

\paragraph{例题} HDU5901

\lstinputlisting{math/number_theory/prime/count_primes.cpp}

\subsection{区间求互质数(容斥原理)}

求$[1,r]$内与$n$互素的数的个数。

\paragraph{注} 需要分解质因数函数get\_factors。

\paragraph{例题} HDU4135

\lstinputlisting{math/number_theory/prime/find_coprime.cpp}

\subsection{Miller Rabin}

不断选取不超过$n-1$的底数$a$,每次判断是否为$a^{n-1}\equiv 1(\mod n)$。如果有一次不成立,则$n$为合数,否则基本上可以断定为素数(仍然有合数的可能)。

\paragraph{注} 需要快速幂算法quick\_pow,以及快速乘法quick\_mul。

\lstinputlisting{math/number_theory/prime/miller_rabin.cpp}

\subsection{反素数表}
\lstinputlisting{math/number_theory/prime/emirps.cpp}
