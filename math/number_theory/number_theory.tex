\subsection{GCD和LCM}

求两个数的最大公约数的基本思想:$\gcd(a,b)=\gcd(b,a\mod b)$

两个数的最大公约数和最小公倍数的关系为 $\gcd(a,b) \times \operatorname{lcm}(a,b)=a \times b$,因此可以通过 $\operatorname{lcm}(a,b) = a \times b \div \gcd(a,b)$ 来求出最小公倍数。

如果求多个数值的最大公约数或最小公倍数,可以先求出前两个数的值,然后再将结果和第三个数一起求值。

\lstinputlisting{math/number_theory/gcd_lcm.cpp}

\subsection{扩展欧几里得}

以下算法用于求$ax+by=\gcd(a,b)$的一组整数解。注意,方程的解不是唯一的,因为使$x$增加若干个$b$,使$y$减少同等数量的$a$,结果还是不变的。

按照求解最大公约数的基本思想,我们可以得到一个式子。与原方程联立,则有

\begin{equation}
  \left\{
    \begin{array}{l}
      ax_{1}+by_{1}=\gcd(a,b) \\
      bx_{2}+(a\%b)y_{2}=\gcd(b,a\%b)
    \end{array}
  \right.
\end{equation}

很明显,等号右面的两个值是相等的,而且有 $a\%b=a-(a/b) \times b$,那么

\begin{equation}
  ax_{1}+by_{1}=bx_{2}+[a-(a/b) \times b] \times y_{2}
\end{equation}

也就是

\begin{equation}
  ax_{1}+by_{1}=ay_{2}+b[x_{2}-(a/b)y_{2}]
\end{equation}

由于等式是恒成立的,所以可以得出

\begin{equation}
  x_{1}=y_{2},\; y_{1}=x_{2}-(a/b) \times y_{2}
\end{equation}

\lstinputlisting{math/number_theory/extended_gcd.cpp}

\subsection{快速幂}

假设$a$,$b$均为整数,且$b>0$。怎样计算$a^b$的值?

当然可以使用<cmath>里的pow函数,不过在取整的时候要注意在结果上加一个小数(比方说pow(2,10),正确答案是$1024$,但是由于C++的浮点数存在误差,所以最终结果可能是$1024.00000001$,也可能是$1023.99999999$。对于后者来说,如果直接取整,那么结果就会变成$1023$),即:

\begin{lstlisting}
int r = int(pow(2,10) + 0.0001);
\end{lstlisting}

我们也可以使用for循环来计算$a^b$,时间复杂度$\mathcal{O}(b)$。显然,这个时间复杂度较高。

实际上我们可以把 $b$ 按照二进制来分解。以 $b=77$ 为例,$77=(1001101)_{2}=2^{6}+2^{3}+2^{2}+2^{0}$。求和的过程可以通过递推来完成,即 $77=(((((1\times 2+0)\times 2+0)\times 2+1)\times 2+1)\times 2+0)\times 2+1$。这样时间复杂度就缩小到 $\mathcal{O}(\log b)$了。

这时,我们就可以通过递推把$a^{1}$,$a^{4}$,$a^{8}$,$a^{64}$算出来,然后将它们组合成结果。

\lstinputlisting{math/number_theory/quick_pow.cpp}

类似的思想还可以用于求$a^{1}+a^{2}+a^{3}+a^{4}+ \cdots +a^{n}$的和,如:

\begin{equation}
  a^{1}+a^{2}+a^{3}+\cdots +a^{{16}}=a^{8}(a^{1}+a^{2}+\cdots +a^{8})+(a^{1}+a^{2}+\cdots +a^{8})
\end{equation}

\subsection{Java大数快速幂}

\begin{lstlisting}
public static BigInteger quickPow(BigInteger x, BigInteger n, BigInteger m) {
  BigInteger ret = BigInteger.ONE;
  while (n.compareTo(BigInteger.ZERO) == 1) {
    if (n.mod(BigInteger.valueOf(2)).equals(BigInteger.ONE)) {
      ret = ret.multiply(x).mod(m);
    }
    x = x.multiply(x).mod(m);
    n = n.divide(BigInteger.valueOf(2));
  }
  return ret;
}
\end{lstlisting}

\subsection{快速乘法}
\lstinputlisting{math/number_theory/quick_mul.cpp}

\subsection{BSGS}

BSGS(Baby Steps Giant Steps),即大小步算法。

\subsubsection{普通版本}

给定素数$P$,$2 \leq P < 2^{31}$,整数$B$,$2 \leq B < P$,和整数$N$,$1 \leq N < P$,求解整数$L$使其满足:

\begin{equation}
  B^L \equiv N \pmod P
\end{equation}

\paragraph{例题} POJ2417

\paragraph{注} 该算法需要HashTable。

\lstinputlisting{math/number_theory/bsgs.cpp}

\subsubsection{扩展版本}

给定整数$B$,$N$,$P$,求解整数$L$使其满足:

\begin{equation}
  B^L \equiv N \pmod P
\end{equation}

\paragraph{例题} POJ3243

\paragraph{注} 该算法需要HashTable,扩展欧几里得函数extended\_gcd,和求解最大公约数函数gcd。

\lstinputlisting{math/number_theory/extended_bsgs.cpp}

\subsection{欧拉函数}
\lstinputlisting{math/number_theory/euler.cpp}

\subsection{欧拉函数打表}

\paragraph{例题} HDU2824

\lstinputlisting{math/number_theory/euler_table.cpp}

\subsection{分解质因数}
\lstinputlisting{math/number_theory/factors.cpp}

\subsection{Pollard Rho}

\paragraph{例题} POJ1811

\paragraph{注} 需要Miller Rabin素数测试函数miller\_rabin,求解最大公约数函数gcd,快速乘函数quick\_mul,以及经过快速乘改写的快速幂函数quick\_pow。

\lstinputlisting{math/number_theory/pollard_rho.cpp}

\subsection{因数个数打表}

打表区间$[1,n]$中所有数的因子的个数。

\lstinputlisting{math/number_theory/count_factors.cpp}

\subsection{线性同余方程}

\paragraph{问题} 如何求$ax \equiv b \pmod n \; (n>0)$的解?如果有解,有几个解?\footnote{假如存在$x_{0}$ 使 $ax_{0} \equiv b \pmod n$成立,那么很明显,$x_{0}+kn \; (k\in \mathbb{Z})$ 也可以使式子成立。在数论中,我们认为这是一个解。}

设$y$是一个整数,那么上述式子就可以转化为$ax-ny=b$,也就是说,如果上面那个方程有解,那么这个二元方程也必须有解。为了利用扩展欧几里得算法,我们要对方程稍作转化。令$d=\gcd(a,n)$,我们先求解$ax-ny=d$(若使方程有解,需要$d$能够整除$b$),可知这个$x$乘上$\frac{b}{d}$就是其中的一个解。

此方程共有$d$个解,这些解相差$\frac{n}{d}$的整数倍。求出一个解之后,其他解也就能求出来了。

\paragraph{注} 该算法需要扩展欧几里得函数extended\_gcd。

\lstinputlisting{math/number_theory/linear_congruence_mod_equation.cpp}

\subsection{中国剩余定理}

\paragraph{问题} 一个数,被3除余1,被5除余2,被7除余3,那么这个数是多少?

\subsubsection{互质情况}

设$m_1, m_2, \cdots, m_k$ \textbf{两两互素},则下面同余方程组

\begin{equation}
  \begin{array}{c}
    x \equiv a_1 \pmod{m_1} \\
    x \equiv a_2 \pmod{m_2} \\
    \vdots \\
    x \equiv a_k \pmod{m_k}
  \end{array}
\end{equation}

令$M=m_1m_2 \cdots m_k$,则方程组在$0 \leq x < M$内有唯一解。

\paragraph{例} POJ1006

\paragraph{注} 该算法需要扩展欧几里得函数extended\_gcd。

\lstinputlisting{math/number_theory/chinese_remainder_theorem_coprime.cpp}

\subsubsection{非互质情况}

先考虑两组方程的情况:

\begin{equation}
  \begin{array}{l}
    x \equiv a_1 \pmod{m_1} \\
    x \equiv a_2 \pmod{m_2}				
  \end{array}
\end{equation}

有解的充要条件是 $\gcd(m_1, m_2) | (a_1-a_2)$。如果有解,先将以上二式转化:

\begin{equation}
  \begin{array}{l}
    x = a_1 + m_1 y	\\
    x = a_2 + m_2 z
  \end{array}
\end{equation}

联立得$a_1 + m_1 y = a_2 + m_2 z$。令$c=\gcd(m_1, m_2)$,则

\begin{equation}
  \frac{m_2}{c}z - \frac{m_1}{c}y = \frac{a_1-a_2}{c}
\end{equation}

然后用扩展的欧几里得算法求出$p$和$q$,使得

\begin{equation}
  \frac{m_2}{c}p + \frac{m_1}{c}q = \gcd\left(\frac{m_2}{c},\frac{m_1}{c}\right)=1
\end{equation}

那么只要取 $z=p\times(a_1-a_2)/c$,则 $x=a_2+m_2z$。

对于$m_i$不是两两互素的情况,可以每次解两个方程,然后将两方程合并。

\paragraph{例题} POJ2891

\paragraph{注} 该算法需要扩展欧几里得函数extended\_gcd。

\lstinputlisting{math/number_theory/chinese_remainder_theorem_noncoprime.cpp}

\subsection{莫比乌斯函数}

\paragraph{例题} GYM101982B

\lstinputlisting{math/number_theory/mobius.cpp}

\subsection{原根}

\paragraph{例题} HDU4992

\paragraph{注} 该算法需要筛法求质数函数sieve\_euler,求解最大公约数函数gcd,欧拉函数euler,快速幂函数quick\_pow,分解质因数函数get\_factors。

\lstinputlisting{math/number_theory/primitive_root.cpp}

\subsection{快速傅里叶变换 (FFT)}
\lstinputlisting{math/number_theory/fft.cpp}

\subsection{快速数论变换 (NTT)}

\paragraph{注} 该算法需要快速幂函数quick\_pow。

\lstinputlisting{math/number_theory/ntt.cpp}
